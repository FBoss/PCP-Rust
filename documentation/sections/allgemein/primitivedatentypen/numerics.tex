\subsubsection{Numerische Datentypen}
Rust unterscheidet zwischen mehreren Kategorien: signed und unsigned, fixed and variable, floating-point and integer.

Der Datentyp besteht aus 2 Teilen. Dem Typ und der Grösse.
Zum Beispiel ist der "u16" Typ ein unsigned integer mit 16 bit Grösse.

Wird bei einer Zahl kein Typ angegeben gilt standardmässig f64 oder i32. Um einen Datentyp explizit anzugeben wird der Datentyp mit einem ":" von der Variablenbezeichnung angegeben.

\begin{lstlisting}
let x = 42; // x has type i32
let y = 1.0; // y has type f64
let z : i8 = 8 // z has type i8
\end{lstlisting}

Folgende primititven Datentypen existieren in Rust: 
\begin{itemize}
\item i8
\item i16
\item i32
\item i64
\item u8
\item u16
\item u32
\item u64
\item isize
\item usize
\item f32
\item f64
\end{itemize}

Fixed Size Typen sind alle die eine Zahl (8,16,32,64) tragen (z.B. i8, f32, u64,...).

Variable Grössen sind "isize" und "usize". Ihre Grösse richtet sich nach der Pointergrösse der darunterliegenden Maschine.

Die Floating Point Datentypen "f32" und "f64" folgen der  IEEE-754 Single und Double Precision Konvention.