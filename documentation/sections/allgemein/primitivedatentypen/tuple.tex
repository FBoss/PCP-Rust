\subsubsection{Tuple}
Ein Tuple ist eine geordnete List fixer Grösse. Stimmt Ihre Signatur (Länge und Datentypen) überein können sie einander zugewiesen werden.
Zugriff auf die einzelnen Elemente kann mit einem Dekonstruktor let erreicht werden.

\begin{lstlisting}
let x = (1, "hello"); // x: [i32, &str]
let y: (i32, &str) = (1, "hello");

let mut a = (1, 2); // a: (i32, i32)
let b = (2, 3); // b: (i32, i32)
a = b;

let (x, y, z) = (1, 2, 3);
println!("x is {}", x);
\end{lstlisting}

Tuple-Elemente können ebenfalls über ihren Index angesprochen werden. Im Gegensatz zu Array wird der Index mit "." angegeben.

\begin{lstlisting}
let tuple = (1, 2, 3);

let x = tuple.0;
let y = tuple.1;
let z = tuple.2;

println!("x is {}", x);
\end{lstlisting}