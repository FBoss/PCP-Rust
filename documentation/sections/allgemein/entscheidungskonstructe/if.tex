\subsubsection{if}
Wie in vielen Programmiersprachen kann in Rust das if, if else, else Konstrukt verwendet werden.

\begin{lstlisting}
let x = 5;

if x == 5 {
	println!("x is five!");
} else if x == 6 {
	println!("x is six!");
} else {
		println!("x is not five or six :(");
}
\end{lstlisting}

Da in Rust dieses Konstrukt eine Expression ist, kann ebenfalls direkt eine Variable initialisiert werden.

\begin{lstlisting}
let x = 5;

let y = if x == 5 { 10 } else { 15 }; // y: i32
\end{lstlisting}