\subsection{Schleifen vorzeitig beenden}
In Rust gibt es 3 Befehle um Schleifen vorzeitig zu beenden.
\begin{itemize}
\item break
\subitem Bricht den Schleifendurchlauf ab.
\item return
\subitem Bricht den Schleifendurchlauf ab und verlässt die aktuelle Funktion.
\item continue
\subitem Bricht den aktuellen Schleifendurchlauf ab und beginnt mit dem nächsten.
\end{itemize}

\begin{lstlisting}
let mut x = 5;
let mut done = false;

while !done {
	x += x - 3;

	println!("{}", x);

	if x % 5 == 0 {
		done = true;
	}
}
\end{lstlisting}

Mit dem break Statement kann die obere while Schleife mit einem infinite loop ersetzt und die done variable gestrichen werden.

\begin{lstlisting}
let mut x = 5;

loop {
	x += x - 3;

	println!("{}", x);

	if x % 5 == 0 { break; }
}
\end{lstlisting}

Schleifen können mit einem Label versehen werden.
So kann falls mehrere Schleifen verschachtelt sind angegeben werden für welche Schleife der Abbruch Befehl ausgeführt werden soll.

\begin{lstlisting}
'outer: for x in 0..10 {
	'inner: for y in 0..10 {
		if x % 2 == 0 { continue 'outer; } // continues the loop over x
		if y % 2 == 0 { continue 'inner; } // continues the loop over y
		println!("x: {}, y: {}", x, y);
	}
}
\end{lstlisting}