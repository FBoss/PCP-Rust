\subsubsection{for-Schleife}
Die For-Schleifen in Rust sind ähnlich wie foreach Schleifen in anderen Programmiersprachen. Sie sollte verwendet werden wenn bekannt ist wie oft geloopet werden soll.
Generell geschrieben sieht die Schleife wiefolgt aus.
\begin{lstlisting}
for var in expression {
code
}
\end{lstlisting}

Im Beispiel unten ist x die Variable welche nacheinander die Werte von 0 bis 9 (expression) erhält. Der Ausdruck 0..10 erzeugt einen Iterator mit Startwert 0 bis 10-1 wobei jeder nachfolgede Wert inkrementiert wird. 
\begin{lstlisting}
for x in 0..10 {
println!("{}", x); // x: i32
}
\end{lstlisting}

Um Nachzuverfolgen wie oft der Loop bereits durchgeführt werden kann einer Enumerator verwendet werden. Im Beispiel unten wird bei jedem Schleifendurchlauf dem Tupel (i,j) in der i Komponente der Durchlaufnummer (beginnend mit 0) und in j der Wert des Durchlaufs zugeweisen.

\begin{lstlisting}
for (i,j) in (5..10).enumerate() {
println!("i = {} and j = {}", i, j);
}
\end{lstlisting}
Output:
\begin{lstlisting}
i = 0 and j = 5
i = 1 and j = 6
i = 2 and j = 7
i = 3 and j = 8
i = 4 and j = 9
\end{lstlisting}