\subsubsection{Strings}
Strings sind eine folge vom Unicode Charactern ecodiert mit UTF-8 und können null Bytes enthalten. Es gibt 2 Verschiedene String Typen in Rust ("\&str" und "String"). Der Unterschied liegt darin das \&str eine Fixe Länge hat und immutable ist. Mit der Methode ".to\_string()" kann er einfach innen String umgewandelt werden String hingegen ist variabel in der Länge und mutable. Die Umwandlung in den Typ \&str kann mit dem \& Operator erreicht werden.  angesprochen werden.

\begin{lstlisting}
let mut s = "Hello".to_string(); // mut s: String
println!("{}", s);

s.push_str(", world.");
println!("{}", s);

Strings will coerce into &str with an &:

fn takes_slice(slice: &str) {
println!("Got: {}", slice);
}

fn main() {
let s = "Hello".to_string();
takes_slice(&s);
}
\end{lstlisting}


Da Strings UTF-8 kodiert sind kann ein Charakter ein oder mehrere Byte lang sein. Deshalb kann ein Charakter nicht über einen Index angesprochen werden. Mit der Methode ".nth(1)" (entspricht ca. [1] bei einem Array) kann ein einzelnes Zeichen angesprochen werden. Dazu wird jedoch der ganze String von Anfang an durchlaufen was Zeitintensiv ist.

\begin{lstlisting}
let hachiko = "Hachiko";

for b in hachiko.as_bytes() {
	print!("{}, ", b);
}

println!("");

for c in hachiko.chars() {
	print!("{}, ", c);
}

println!("");
let dog = hachiko.chars().nth(1); // kinda like hachiko[1]
\end{lstlisting}


Ein \&str kann am Ende eines String angehängt werden.

\begin{lstlisting}
let hello = "Hello ".to_string();
let world = "world!";

let hello_world = hello + world;
\end{lstlisting}

Um einen String am Ende eines String anzuhägen muss dieser zuerst in ein \&str umgewandelt werden.

\begin{lstlisting}
let hello = "Hello ".to_string();
let world = "world!".to_string();

let hello_world = hello + &world;
\end{lstlisting}