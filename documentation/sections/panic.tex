\section{Panics}

Mit "panic!" kann ein Thread gecrashed werden.
Nach dem join kann dann überprüft werden ob der Thread eine panic ausgelöst hat oder nicht. In Java entspricht dies etwa "throw new RunTimeException"

\begin{lstlisting}
use std::thread;

let handle = thread::spawn(move || {
panic!("oops!");
});

let result = handle.join();

assert!(result.is_err());
\end{lstlisting}