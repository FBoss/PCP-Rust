\section{Matching}
In einer Applikation ist es oft nicht ausreichend ein einfaches \textcolor{blue}{if}/\textcolor{purple}{else}
Konstrukt zu benutzen. Wenn man zum Beispiel mehr als nur zwei Optionen hat oder die Bedingungen sehr komplex werden. Für solche Fälle hat Rust das Schlüsselwort \textbf{match}.

\begin{lstlisting}
let x = 5;

match x {
    1 => println!("one"),
    2 => println!("two"),
    3 => println!("three"),
    4 => println!("four"),
    5 => println!("five"),
    _ => println!("something else"),
}
\end{lstlisting}
Mit \textbf{match} wird der Wert mit allen Statements verglichen. Trifft eines zu wird der Programmierarm dahinter ausgeführt. Als letztes Statement jedes \textbf{match}es muss das Underscore Pattern \_ stehen. Damit ein \textbf{match} vollständig ist. Ohne dieses Pattern würde das Programm nicht compilieren.